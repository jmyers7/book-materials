\documentclass[12pt,reqno]{amsart}
\usepackage{./header}

\hdr{Mathematical Statistics}{Chapter 3: Random variables, \textit{etc}.}

\begin{document}

\bigskip

\prob Let the sample space $S$ represent the current population of the United States. Suppose also that to every individual $s\in S$ we assign the state $X(s)$ in which they currently reside. So, for example, I would write

	\[
	X(\text{John}) = \text{New York}.
	\]

Does this function

	\[
	X: S \to \{\text{Alabama}, \text{Alaska},\ldots,\text{Wyoming}\}
	\]

qualify as a random variable on $S$? If not, how could you make it fit the template of a random variable?

\bigskip
\textcolor{red}{\textit{No}, this is not a valid random variable since random variables are supposed to output numbers. We could change $X$ into a random variable by assigning a number to each state. While this would technically work, it might be undesirable since it introduces an artificial order on the states.}
\bigskip
















\bigskip
\prob Suppose that we flip a fair coin twice. As usual, we model the situation using a uniform probability space with sample space

	\[
	S = \{ HH, HT, TH, TT\}.
	\]

Describe some random variables on $S$.

\bigskip
\textcolor{red}{We could let $X$ be a random variable that represents the number of heads that we obtain. Or, $X$ could be the number of tails.}
\bigskip

















\bigskip

\prob Suppose that we toss a pair of fair six-sided dice. As usual, we model the situation using a uniform probability space with sample space

	\[
	S = \{(1,1),(1,2),\ldots,(5,6),(6,6)\}.
	\]

Describe some random variables on $S$.

\bigskip
\textcolor{red}{We could let $X$ be a random variable that represents the first number that we obtain. Or, it could represent the second number. Or, it could be the \textit{sum} of the two numbers.}
\bigskip























\bigskip
\prob Let

	\[
	S = \{ (x,y) \in \mathbb{R}^2 : x^2+y^2 \leq 1\}
	\]

be the solid unit disk in $\mathbb{R}^2$. Consider drawing a point uniformly at random from $S$.

\medskip
\begin{enumerate}
\item Define an appropriate probability space that models this scenario.

\bigskip
\textcolor{red}{The sample space is the disk $S$. The probability measure is continuous and uniform, with density function given by
	\[
	f(x,y) = \frac{1}{\pi}
	\]
for all $(x,y)\in S$.}
\bigskip

\item Describe some random variables in this scenario.

\bigskip
\textcolor{red}{This is similar to the previous problem. We could let $X$ be a random variable that represents the $x$-value that we obtain. Or, it could represent the $y$-value. Or, it could be the \textit{sum} of the $x$- and $y$-values.}
\bigskip
\end{enumerate}

























\bigskip
\prob You and a friend play a game where you each toss a fair coin. If both coins land tails, you win $\$1$; if they both land heads, you win $\$2$; if the coins do not match (one lands a head, the other a tail), you lose $\$1$ (win $-\$1$).

\medskip
\begin{enumerate}
\item Describe an appropriate probability space that models the situation.
    
\bigskip
\textcolor{red}{As usual, the sample space is
    	\[
	S = \{HH,HT,TH,TT\}
	\]
The probability measure is uniform, with the probability of any simple event occurring equal to $1/4$.}
\bigskip

\item Let $X$ be the random variable which gives your winnings on a single play of the game. Describe the probability distribution of $X$.
    
\bigskip
\textcolor{red}{We have
	\begin{align*}
	P(X=1) &= 1/4, \\
	P(X=2) &= 1/4, \\
	P(X=-1) &= 1/2.
	\end{align*}}
\end{enumerate}























\bigskip
\prob Five balls, numbered $1$, $2$, $3$, $4$, and $5$, are placed in an urn. Two balls are randomly selected from the five, and their numbers noted. (The order of the selected balls does not matter.)

\medskip
\begin{enumerate}
\item Describe an appropriate probability space that models the situation.
    
\bigskip
\textcolor{red}{The sample space $S$ is all two-element combinations chosen from $\{1,2,3,4,5\}$. The probability measure is uniform, and since the cardinality of $S$ is $\binom{5}{2} = 10$, the probability of any simple event is $1/10$.}
\bigskip

\item Let $X = $ the largest of the two numbers. Describe the probability distribution of $X$.
    
\bigskip
\textcolor{red}{We have
	\begin{align*}
	P(X=2) &= 1/10, \\
	P(X=3) &= 2/10, \\
	P(X=4) &= 3/10, \\
	P(X=5) &= 4/10.
	\end{align*}}
\bigskip

\item Let $Y = $ the sum of the two numbers. Describe the probability distribution of $Y$.
    
\bigskip
\textcolor{red}{We have
	\begin{align*}
	P(Y=3) &= 1/10, \\
	P(Y=4) &= 1/10, \\
	P(Y=5) &= 2/10, \\
	P(Y=6) &= 2/10, \\
	P(Y=7) &= 2/10, \\
	P(Y=8) &= 1/10, \\
	P(Y=9) &= 1/10.
	\end{align*}}
\end{enumerate}















\bigskip
\prob Let $S$ be the discrete probability space $S = \{1,2,3,4\}$ with probability function

	\[
	\begin{array}{c|c}
	s & p(s) \\ \hline
	1 & 1/3\\
	2 & 1/3\\
	3 & 1/6\\
	4 & 1/6
	\end{array}
	\]

Define the random variable $X:S\to \mathbb{R}$ by

	\[
	X(s) = \begin{cases}
	0 & : s = 1, 4, \\
	1 & : s=2, \\
	2 & : s=3.
	\end{cases}
	\]

Describe the probability distribution of $X$.

\bigskip
\textcolor{red}{We have
	\begin{align*}
	P(X=0) &= 1/3 + 1/6 = 1/2, \\
	P(X=1) &= 1/3, \\
	P(X=2) &= 1/6.
	\end{align*}}










\bigskip
\prob Which of the random variables in problems 5-7 are discrete? Which are continuous?

\bigskip
\textcolor{red}{They are \textit{all} discrete, since they all have ranges that contain only finitely many values!}













\bigskip
\prob A gas station operates two pumps, each of which can pump up to 10,000 gallons of gas in a month. The total amount of gas pumped at the station in a month is a continuous random variable $X$ (measured in 10,000 gallons) with a probability density function given by

	\[
	f(x) = \begin{cases}
	x & : 0 < x < 1, \\
	2-x & : 1 \leq x < 2, \\
	0 & : \text{otherwise}.
	\end{cases}
	\]

\medskip
\begin{enumerate}
\item Find the probability that the station will pump between 8000 and 12,000 gallons in a particular month.
    
\bigskip
\textcolor{red}{We have
    	\[
	P(0.8 \leq X \leq 1.2) = \int_{0.8}^{1.2}f(x) \ \text{d} x = \int_{0.8}^1 x \ \text{d} x + \int_1^{1.2}(2-x) \ \text{d} x = 0.36.
	\]}
\bigskip

\item Given that the station pumped more than 10,000 gallons in a particular month, find the probability that the station pumped more than 15,000 gallons during a month.
    
\bigskip
\textcolor{red}{We are asked to compute the conditional probability
    	\[
	P(X\geq 1.5 \mid X\geq 1) = \frac{P(X\geq 1.5 \text{ and } X\geq 1)}{P(X\geq 1)} = \frac{P(X\geq 1.5)}{P(X\geq 1)}.
	\]
    But we have
    	\[
	P(X\geq 1) = \int_1^\infty f(x) \ \text{d}x = \int_1^2(2-x) \ \text{d} x = 1/2
	\]
    and
    	\[
	P(X\geq 1.5) = \int_{1.5}^\infty f(x) \ \text{d} x = \int_{1.5}^2 (2-x) \ \text{d} x = 1/8.
	\]
    Hence
    	\[
	P(X\geq 1.5 \mid x\geq 1) = 1/4.
	\]}
\end{enumerate}









\bigskip
\prob The length of time to failure (in hundreds of hours) for a transistor is a random variable $X$ with
distribution function given by

	\[
	F(x) = \begin{cases}
        1 - e^{-x^2} & : x\geq 0, \\
        0 & : \text{otherwise}.
	\end{cases}
	\]

\medskip
\begin{enumerate}
\item Compute the quantile $Q(0.3)$.
    
\bigskip
\textcolor{red}{We compute the value $x = Q(0.3)$ by solving $F(x) = 0.3$ for $x$:
    	\[
	F(x) = 0.3 \quad \Rightarrow \quad 1 - e^{-x^2} = 0.3 \quad \Rightarrow \quad x = \sqrt{-\ln{0.7}} \approx 0.60.
	\]}
\bigskip

\item Compute the density $f(x)$ of $X$.
    
\bigskip
\textcolor{red}{We use the Fundamental Theorem of Calculus to compute:
    	\[
	f(x) = F'(x) = \begin{cases}
	2xe^{-x^2} & : x>0, \\
	0 & : x<0.
	\end{cases}
	\]}
\bigskip

\item Find the probability that the transistor operates for at least 200 hours.
    
\bigskip
\textcolor{red}{We compute:
    	\[
	P(X\geq 2) = 1 - P(X<2) = 1- F(2) \approx 0.02.
	\]}
\bigskip

\item Compute the conditional probability $P(X>1 \mid X\leq 2)$.
    
\bigskip
\textcolor{red}{We compute:
    	\[
	P(X>1 \mid X\leq 2) = \frac{P(X>1 \text{ and } X\leq 2)}{P(X\leq 2)} = \frac{F(2)-F(1)}{F(2)} \approx 0.36.
	\]}
\bigskip
\end{enumerate}



















\bigskip
\prob Compute the expected value $E(X)$ of the random variable $X$ in problem 5.

\bigskip
\textcolor{red}{We have:
	\[
	E(Y) = 1\left( \frac{1}{4} \right) + 2\left( \frac{1}{4} \right) - 1 \left( \frac{1}{2} \right) = 0.25.
	\]}

























\bigskip
\prob Suppose $X$ is a discrete random variable distributed uniformly on its range
    
	\[
	\{1,2,\ldots,n\},
	\]

for some integer $n\geq 1$. Compute the mean value of $X$.

\bigskip
\textcolor{red}{Since $X$ is distributed uniformly, its probability function is $p(x) = 1/n$ for all $x$ in the range. Thus:
	\[
	\mu_X = \sum_{x=1}^n x\cdot \frac{1}{n} = \frac{n(n+1)}{2} \cdot \frac{1}{n} = \frac{n+1}{2}.
	\]
On the other hand, if we think of $\mu_X$ as the ``center of mass'' of the probability distribution, then $\mu_X$ should be halfway between $1$ and $n$. But the midpoint between these two numbers is precisely their average:
	\[
	\mu_X = \frac{n+1}{2}.
	\]}


\bigskip
\prob Let $X$ be the number of interviews that a student has prior to getting a job. Suppose that the probability function of $X$ is given by

	\[
	p(x) = \begin{cases}
	k/x^2 & : x=1,2,\ldots, \\
	0 & : \text{otherwise,}
	\end{cases}
	\]

where $k$ is a constant such that $\sum_{x=1}^\infty k/x^2 = 1$. Compute the mean $\mu_X$ (if it exists).

\bigskip
\textcolor{red}{By definition, we have:
	\[
	\mu_X = \sum_{x=1}^\infty x\cdot  \frac{k}{x^2} = k\cdot \sum_{x=1}^\infty \frac{1}{x}.
	\]
However, the series on the right is the infamous harmonic series, which---as you well know from your training in calculus---diverges. Thus, $\mu_X$ does not exist.}

















\bigskip
\prob Let $X$ be a continuous random variable with density function
	\[
	f(x) = \begin{cases}
	(3/8)x^2 & : 0 \leq x \leq 2, \\
	0 & : \text{otherwise}.
	\end{cases}
	\]

Compute the mean value of $X$.

\bigskip
\textcolor{red}{We have
	\[
	E(X) = \int_{-\infty}^\infty x \cdot f(x) \ \text{d}x = \int_0^2 (3/8)x^3 \ \text{d} x = 3/2.
	\]}












\bigskip
\prob The proportion of time per day that all checkout counters in a supermarket are busy is a random variable $X$ with density function

	\[
	f(x) = \begin{cases}
	cx^2(1-x)^4 & : 0 \leq x \leq 1, \\
	0 & : \text{otherwise}.
	\end{cases}
	\]

\medskip
\begin{enumerate}
\item Find the value of $c$ that makes $f(x)$ a valid density function.
    
\bigskip
\textcolor{red}{We must solve the equation
    	\[
	\int_{-\infty}^\infty f(x) \ \text{d} x  = 1
	\]
for $c$. But
    	\[
	\int_{-\infty}^\infty f(x) \ \text{d} x = \int_0^1 cx^2(1-x)^4 \ \text{d} x = \frac{c}{105},
	\]
and hence we must have $c=105$.}
\bigskip

\item Compute the expected value $E(X)$.
    
\bigskip
\textcolor{red}{We compute:
    	\[
	E(X) = \int_{-\infty}^\infty x \ \cdot f(x) \ \text{d} x = 105 \int_0^1 x^3(1-x)^4 \ \text{d} x = 3/8.
	\]}
\end{enumerate}















\bigskip
\prob Consider the random variables $X$ and $Y$ defined in lecture via the table:

	\[
	\begin{array}{c|ccc}
	s & X(s) & Y(s) & (X+Y)(s) \\ \hline
	1 & -1 & 0 & -1 \\
	2 & 1 & 2 & 3 \\
	3 & 3 & -1 & 2 \\
	4 & 0 & 3 & 3
	\end{array}
	\]

Suppose that the probability distribution on $S$ is uniform.

\medskip
\begin{enumerate}
\item Compute the probability distributions of $X$, $Y$, and $X+Y$.

\bigskip
\textcolor{red}{
For $X$, we have:
	\begin{align*}
	P(X=-1) &= 1/4, \\
	P(X=0) &= 1/4, \\
	P(X=1) &= 1/4, \\
	P(X=3) &= 1/4.
        \end{align*}
For $Y$, we have:
	\begin{align*}
	P(Y = -1) &= 1/4, \\
	P(Y = 0) &= 1/4, \\
	P(Y = 2) &= 1/4, \\
	P(Y=3) &= 1/4.
	\end{align*}
For $X+Y$, we have:
	\begin{align*}
	P(X+Y = -1) &= 1/4, \\
	P(X+Y = 2) &= 1/4, \\
	P(X+Y = 3) &= 1/2.
	\end{align*}
   In particular, notice that both $X$ and $Y$ are distributed uniformly on their ranges, but $X+Y$ is not!}
\bigskip

\item Without me even telling you, I bet you can guess the definition of the pointwise product $XY$. List the outputs of $XY$, and compute its probability distribution.
    
\bigskip
\textcolor{red}{The pointwise product is just the row-wise product:
    	\[
	\begin{array}{c|ccc}
	s & X(s) & Y(s) & (XY)(s) \\ \hline
	1 & -1 & 0 & 0 \\
	2 & 1 & 2 & 2 \\
	3 & 3 & -1 & -3 \\
	4 & 0 & 3 & 0
	\end{array}
	\]
Then:
	\begin{align*}
	P(XY = -3) &= 1/4, \\
	P(XY = 0) &= 1/2, \\
	P(XY = 2) &= 1/4.
	\end{align*}}
\end{enumerate}
    











\bigskip
\prob Suppose that we have a random variable $X$ on the finite sample space $S = \{1,2,3,4,5\}$ with

	\[
	\begin{array}{c|c}
	s & X(s)  \\ \hline
	1 & 0  \\
	2 & \pi/2  \\
	3 & \pi \\
	4 & 3\pi/2 \\
	5 & 2\pi
	\end{array}
	\]

\medskip
\begin{enumerate}
\item Compute the random variable $\sin{(X)}$.
    
\bigskip
\textcolor{red}{We have:
    	\[
	\begin{array}{c|cc}
	s & X(s) & \sin(X)(s)  \\ \hline
	1 & 0 & 0  \\
	2 & \pi/2 & 1  \\
	3 & \pi & 0 \\
	4 & 3\pi/2 & -1 \\
	5 & 2\pi & 0
	\end{array}
	\]}
\bigskip

\item Assume that the (discrete) probability distribution on $S$ has probability function $p(s)$ with
    	\[
	\begin{array}{c|c}
	s & p(s)  \\ \hline
	1 & 0  \\
	2 & 1/8  \\
	3 & 1/2 \\
	4 & 1/4 \\
	5 & 1/8
	\end{array}
	\]
Compute the probability distribution of $\sin(X)$.

\bigskip
\textcolor{red}{We have:
	\begin{align*}
	P(\sin(X) = -1) &= 1/4, \\
	P(\sin(X) = 0) &= 5/8, \\
	P(\sin(X) = 1) &= 1/8. \\
	\end{align*}}
\end{enumerate}
















\bigskip
\prob Compute the expectation of the random variable $\sin(X)$ in the previous problem.

\bigskip
\textcolor{red}{Let's do this two ways. First, by using the definition of the expectation,
	\[
	E(\sin{X}) = \sum_{y\in \mathbb{R}} y\cdot  p_{\sin{X}}(y),
	\]
where $p_{\sin{X}}(y)$ is the mass function of the random variable $Y=\sin{X}$. Using the results of the previous problem, we get:
	\[
	E(\sin(X)) = 1 \cdot( 1/8) -1 \cdot (1/4) = -1/8.
	\]
On the other hand, we can also use the LotUS:
	\[E(\sin{X}) = \sum_{x\in \mathbb{R}} \sin{(x)}  \cdot p_X(x).
	\]
Now, we compute:
	\[
	\begin{array}{c|c}
	x & p_X(x)  \\ \hline
	0 & 0  \\
	\pi/2 & 1/8  \\
	\pi & 1/2 \\
	3\pi/2 & 1/4 \\
	2\pi & 1/8
	\end{array}
	\]
Then, using the previously displayed formula, we get:
	\begin{align*}
	E(\sin{X}) &= \sin(0) \cdot 0 + \sin(\pi/2) \cdot (1/8) + \sin(\pi) \cdot (1/2) + \sin(3\pi/2) \cdot (1/4) + \sin(2\pi) \cdot (1/8) \\
	&= 1/8 - 1/4 \\
	&= -1/8.
	\end{align*}}











\bigskip
\prob Let $X$ be a continuous random variable with density function
	\[
	f(x) = \begin{cases}
	(3/8)x^2 & : 0 \leq x \leq 2, \\
	0 & : \text{otherwise}.
	\end{cases}
	\]

Compute the mean value of $2X^2+1$.

\bigskip
\textcolor{red}{By the LotUS with $g(x) = 2x^2+1$, we have
	\[
	E(2X^2+1) = \int_{-\infty}^\infty(2x^2+1)f(x) \ \text{d} x = \frac{3}{8} \int_0^2 (2x^2+1)x^2 \ \text{d} x = 5.8.
	\]}













\bigskip
\prob Let $X$ be a random variable on the same sample space $S = \{1,2,3,4\}$ with the following values:

	\[
	\begin{array}{c|c}
	s & X(s)  \\ \hline
	1 & 0  \\
	2 & 1  \\
	3 & 0 \\
	4 & 3
	\end{array}
	\]

Supposing that the probability distribution on $S$ has probability function given by

	\[
	\begin{array}{c|c}
	s & p(s)  \\ \hline
	1 & 1/3  \\
	2 & 1/9  \\
	3 & 1/3 \\
	4 & 2/9
	\end{array}
	\]

compute the expectation of the random variable $X^2 + X + 2$.

\bigskip
\textcolor{red}{We will do this in two ways: First, I want to use ``weak'' linearity of expectations to compute the desired expectation like this:
	\[
	E(X^2 + X + 2) = E(X^2) + E(X) + E(2).
	\]
Of course, we know that $E(2)=2$ since $2$ is a \textit{constant} random variable. As for the other two expectations $E(X)$ and $E(X^2)$, we first compute:
	\[
	\begin{array}{c|cc}
	s & X(s) & X^2(s)  \\ \hline
	1 & 0 & 0  \\
	2 & 1 & 1  \\
	3 & 0 & 0 \\
	4 & 3 & 9
	\end{array}
	\]
Then we have
	\[
	E(X) = 1\cdot (1/9) + 3\cdot(2/9) \quad \text{and} \quad E(X^2) = 1 \cdot (1/9) + 9 \cdot (2/9),
	\]
so
	\[
	E(X^2 + X +2) = \frac{44}{9} \approx 4.89.
	\]
However, we could have also computed the expectation by using the LotUS with the function $g(x) = x^2 + x + 2$. In this case, we would compute:
	\[
	\begin{array}{c|cc}
	s & X(s) & (X^2 + X +2)(s)  \\ \hline
	1 & 0 & 2  \\
	2 & 1 & 4  \\
	3 & 0 & 2 \\
	4 & 3 & 14
	\end{array}
	\]
Then, the LotUS gives us
	\[
	E(X^2 + X + 2) = 2\cdot (1/3 + 1/3) + 4\cdot (1/9) + 14\cdot (2/9) = \frac{44}{9} \approx 4.89,
	\]
which matches our first answer.}















\bigskip
\prob Suppose that $X$ is a discrete random variable with probability function

	\[
	\begin{array}{c|c}
	x & p(x) \\ \hline
	0 & 1/8   \\
	1 & 1/4   \\
	2 & 3/8  \\
	3 & 1/4 
	\end{array}
	\]

Compute the expectation, variance, and standard deviation of $X$.

\bigskip
\textcolor{red}{We compute:
	\[
	\mu = 0\cdot(1/8) + 1 \cdot(1/4) + 2\cdot(3/8) + 3\cdot(1/4) = 7/4 = 1.75,
	\]
and
	\[
	\sigma^2 = (0-1.75)^2(1/8) + (1-1.75)^2(1/4) + (2-1.75)^2(3/8) + (3-1.75)^2(1/4) = 15/16 = 0.9375,
	\]
and
	\[
	\sigma = \sqrt{15/16} \approx 0.97.
	\]}



















\bigskip
\prob Suppose that a single fair six-sided die is tossed once and we let $X$ be the number facing up. Find the expected value, variance, and standard deviation of $X$.

\bigskip
\textcolor{red}{The random variable $X$ is distributed uniformly on its range with $p(x) = 1/6$ for all $x=1,2,3,4,5,6$. Then
	\[
	E(X) = \frac{1}{6} \cdot (1+2+3+4+5+6) = 3.5,
	\]
and
	\[
	V(X) = \frac{1}{6}\cdot \left[ (1-3.5)^2 + (2-3.5)^2 + (3-3.5)^2 + (4-3.5)^2 + (5-3.5)^2 + (6-3.5)^2\right] \approx 2.92,
	\]
and
	\[
	\sigma \approx \sqrt{2.92} \approx 1.71.
	\]}



















\bigskip
\prob Suppose that $X$ is a continuous random variable with density function
	
	\[
	f(x) = \begin{cases}
	(3/2)x^2 + x & : 0\leq x \leq 1, \\
	0 & : \text{otherwise}.
	\end{cases}
	\]

Compute the expectation, variance, and standard deviation of $X$.

\bigskip
\textcolor{red}{We compute:
	\[
	\mu = \int_{-\infty}^\infty x\cdot f(x) \ \text{d} x = \int_0^1 x\big((3/2)x^2+x \big) \ \text{d} x = 17/24 \approx 0.71,
	\]
and
	\[
	V(X) = \int_{-\infty}^\infty (x-\mu)^2f(x) \ \text{d} x = \int_0^1 (x-17/24)^2 \big( (3/2)x^2+x\big) \ \text{d}x \approx 0.05,
	\]
and
	\[
	\sigma \approx \sqrt{0.05} \approx 0.22.
	\]}












\bigskip
\prob

\begin{enumerate}
\item For a certain random variable $X$ it is known that $E(X)=2$ and $V(X) = 3$. What is $E(X^2)$?
    
\bigskip
\textcolor{red}{From the shortcut formula for variance, we get:
    	\[
	E(X^2) = V(X) + E(X)^2 = 3 + 2^2 = 7.
	\]}
\bigskip


\item Let $X$ be a random variable with $E(X)=2$ and $V(X)=4$. Compute the expectation and variance of $3-2X$.
    
\bigskip
\textcolor{red}{Using ``weak'' linearity of expectations, we get
    	\[
	E(3-2X) = E(3) -2 E(X) = 3 - 2\cdot 2 = -1.
	\]
    For the variance, notice that $3-2X$ is an affine transformation of $X$. Then:
    	\[
	V(3-2X) = (-2)^2 V(X) = 4\cdot 4 = 16.
	\]}
\end{enumerate}















\bigskip
\prob Approximately $10\%$ of the glass bottles coming off a production line have serious flaws in the glass. If two bottles are randomly selected, find the mean, variance, and standard deviation of the number of bottles that have serious flaws.

\bigskip
\textcolor{red}{Let $X$ be the number of bottles that have serious flaws. The probability mass function of $X$ is
	\[
	\begin{array}{c|c}
	x & p(x) \\ \hline
	0 & (0.9)^2 = 0.81   \\
	1 & 2(0.1)(0.9) = 0.18   \\
	2 & (0.1)^2 = 0.01
	\end{array}
	\]
Given this, we compute
	\[
	\mu_X = 0\cdot(0.81) + 1\cdot(0.18) + 2\cdot(0.01) = 0.2,
	\]
and
	\[
	\sigma^2_X = E(X)^2 - \mu^2_X = 0^2 \cdot(0.81)+ 1^2\cdot (0.18) + 2^2 \cdot(0.01) - (0.2)^2 = 0.18,
	\]
and
	\[
	\sigma_X = \sqrt{0.18} \approx 0.42.
	\]}
\bigskip


\end{document}