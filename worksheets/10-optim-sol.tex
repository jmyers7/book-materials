\documentclass[12pt,reqno]{amsart}
\usepackage{./header, amssymb}

\hdr{Mathematical Statistics}{Chapter 11: Optimization}

\begin{document}

\bigskip

\prob Write down the gradient descent update rule for the objective function
	\[J(\theta) = \theta^4 - 6\theta^3 + 11 \theta^2 - 7\theta + 4
	\]
from class. Suppose the learning rate is $\alpha$.

\bigskip
\textcolor{red}{The update rule is
	\[\theta := \theta - \alpha J'(\theta) = \theta - \alpha(4\theta^3 - 18\theta^2 +22 \theta -7).
	\]
The recurrence relation is
	\[\theta_{t+1} = \theta_{t} - \alpha J'(\theta_{t}) = \theta_t - \alpha(4\theta_t^3 - 18\theta_t^2 +22 \theta_t -7),
	\]
for $t\geq 0$.}
\bigskip













\prob Consider the affine objective function
	\[J(\theta) = m\theta + b,
	\]
for some parameters $m\neq 0$ and $b\in \mathbb{R}$.

\medskip
\begin{enumerate}
\item Write down the gradient descent update rule. Suppose the learning rate is $\alpha$.

\bigskip
\textcolor{red}{The update rule is
	\[\theta := \theta - \alpha m.
	\]
The recurrence relation is
	\[\theta_{t+1} = \theta_{t} - \alpha m,
	\]
for $t\geq 0$.}
\bigskip

\item Find a closed form expression for $\theta_{t}$.

\bigskip
\textcolor{red}{We have
	\[\theta_{t} = \theta_0 -\alpha mt
	\]
for all $t\geq 1$.}
\bigskip

\item Using your answer to (b), discuss convergence of the gradient descent algorithm.

\bigskip
\textcolor{red}{We have $\theta_t \to -\infty$ as $t\to \infty$.}
\bigskip
\end{enumerate}
















\prob Consider the quadratic objective function
	\[J(\theta) = \theta^2.
	\]

\begin{enumerate}
\item Write down the gradient descent update rule. Suppose the learning rate is $\alpha$.

\bigskip
\textcolor{red}{The update rule is
	\[\theta := \theta - 2\alpha \theta.
	\]
The recurrence relation is
	\[\theta_{t+1} = \theta_{t} - 2\alpha\theta_{t},
	\]
for $t\geq 0$.}
\bigskip

\item Find a closed form expression for $\theta_t$.

\bigskip
\textcolor{red}{We have
	\[\theta_t = (1-2\alpha)^t \theta_0.
	\]}
\bigskip

\item Using your answer to (b), discuss convergence of the gradient descent algorithm.

\bigskip
\textcolor{red}{We have $\theta_t \to 0$ exponentially fast provided $|1-2\alpha|<1$, which occurs if and only if $\alpha < 1$; the value $\theta_t$ orbits back and forth between $-\theta_0$ and $+\theta_0$ if $\alpha =1$; the algorithm diverges to $\infty$ if $\alpha>1$.}
\bigskip
\end{enumerate}










\prob Consider again the affine objective function
	\[J(\theta) = m\theta + b,
	\]
for some parameters $m\neq 0$ and $b\in \mathbb{R}$.

\medskip
\begin{enumerate}
\item Write down the gradient descent update rule with learning rate $\alpha$ and decay rate $\beta$.

\bigskip
\textcolor{red}{The $t$-th update rule is
	\[\theta := \theta - \alpha m(1-\beta)^{t+1}.
	\]
The recurrence relation is
	\[\theta_{t+1} = \theta_{t} - \alpha m(1-\beta)^{t+1},
	\]
for $t\geq 0$.}
\bigskip

\item Find a closed form expression for $\theta_t$.

\bigskip
\textcolor{red}{Setting $\gamma = 1-\beta$ for convenience, we have
	\[\theta_t = \theta_0 - \alpha m \sum_{k=1}^t \gamma^k,
	\]
for $t\geq 1$. But
	\[\sum_{k=1}^t \gamma^k = \frac{\gamma - \gamma^{t+1}}{1-\gamma},
	\]
and so
	\[\theta_t = \theta_0 - \alpha m \left( \frac{\gamma - \gamma^{t+1}}{1-\gamma}\right),
	\]
for $t\geq 1$.}
\bigskip

\item Using your answer to (b), discuss convergence of the gradient descent algorithm.

\bigskip
\textcolor{red}{As we saw in Problem 2, the algorithm diverges if $\beta=0$. But if $\beta>0$, then $\gamma <1$ and
	\[\lim_{t\to \infty} \theta_t = \theta_0 - \alpha m \left( \frac{\gamma}{1-\gamma} \right) = \theta_0 - \alpha m \left( \frac{1-\beta}{\beta} \right).
	\]
Thus, the algorithm converges (but not to a minimizer!) if the decay rate $\beta$ is positive.}
\bigskip
\end{enumerate}











\prob Consider the function
	\[J: \mathbb{R}^2 \to \mathbb{R}, \quad J(\boldsymbol{\theta}) = 2\theta_1^2 + 4\theta_1\theta_2 + 3\theta_2^2 -8\theta_1 - 10\theta_2 + 9.
	\]

\begin{enumerate}
\item Compute the directional first derivative $J'_\mathbf{v}(1,1)$ where $\mathbf{v}^\intercal = (1,0)$.

\bigskip
\textcolor{red}{We compute:
\begin{align*}
J_\mathbf{v}'(1,1) &= \frac{\text{d}}{\text{d}t} \Big|_{t=0} J(1+t,1) \\
&= \frac{\text{d}}{\text{d}t} \Big|_{t=0}\left[ 2(1+t)^2 -4(1+t)+2 \right] \\
&= \left[4(1+t) - 4 \right] \big|_{t=0} \\
&= 0
\end{align*}}
\bigskip

\item Compute the directional second derivative $J''_{\mathbf{v}}(1,1)$ where $\mathbf{v}^\intercal = (1,0)$.

\bigskip
\textcolor{red}{We compute:
\begin{align*}
J_\mathbf{v}''(1,1) &= \frac{\text{d}^2}{\text{d}t^2} \Big|_{t=0} J(1+t,1) \\
&= \frac{\text{d}^2}{\text{d}t^2} \Big|_{t=0}\left[ 2(1+t)^2 -4(1+t)+2 \right] \\
&= 4 \big|_{t=0} \\
&= 4
\end{align*}}
\bigskip
\end{enumerate}








\prob Re-do the previous problem, but use the relationship between directional first and second derivatives and gradient vectors and Hessian matrices.

\bigskip
\textcolor{red}{Let's first compute the gradient vector and Hessian matrix:
	\[\nabla J(\boldsymbol{\theta}) = \begin{bmatrix}
	4\theta_1 + 4\theta_2 -8 \\
	6\theta_2 + 4\theta_1-10
	\end{bmatrix}, \quad
	\nabla^2 J(\boldsymbol{\theta}) = \begin{bmatrix}
	4 & 4 \\
	4 & 6
	\end{bmatrix}.
	\]
Then:
	\[J'_\mathbf{v}(1,1) = \mathbf{v}^\intercal \nabla J(1,1) =  \begin{bmatrix} 1 & 0 \end{bmatrix} \begin{bmatrix} 0 \\ 0 \end{bmatrix} = 0
	\]
and
	\[J''_\mathbf{v}(1,1) = \mathbf{v}^\intercal \left(\nabla^2 J(1,1) \right) \mathbf{v} = \begin{bmatrix} 1 & 0 \end{bmatrix} \begin{bmatrix}
	4 & 4 \\
	4 & 6
	\end{bmatrix} \begin{bmatrix} 1 \\ 0 \end{bmatrix} = \begin{bmatrix} 4 & 4 \end{bmatrix} \begin{bmatrix} 1 \\ 0 \end{bmatrix} = 4.
	\]}
\bigskip







\prob Consider again the function $J$ from Problem 5. Here it is, for reference:
	\[J: \mathbb{R}^2 \to \mathbb{R}, \quad J(\boldsymbol{\theta}) = 2\theta_1^2 + 4\theta_1\theta_2 + 3\theta_2^2 -8\theta_1 - 10\theta_2 + 9.
	\]


\begin{enumerate}
\item Find the direction of maximum rate of change of $J$ at the point $\boldsymbol{\theta}^\intercal = (0,0)$. What is the rate of change in this direction?

\bigskip
\textcolor{red}{Theorem 11.2 tells us that the direction of maximum rate of change is in the direction of the gradient vector
	\[\nabla J(0,0) = \begin{bmatrix}
	-8 \\ -10
	\end{bmatrix}.
	\]
The rate of change in this direction is given by the directional first derivative $J'_\mathbf{v}(0,0)$, where $\mathbf{v}$ is the unit vector that points in the same direction as the gradient:
	\[\mathbf{v} = \frac{\nabla J(0,0)}{|\nabla J(0,0)|}.
	\]
But then
	\[J'_\mathbf{v}(0,0) = \mathbf{v}^\intercal \nabla J(0,0) = \frac{\nabla J(0,0)^\intercal \nabla J(0,0)}{|\nabla J(0,0)|} = \frac{|\nabla J(0,0)|^2}{|\nabla J(0,0)|} = |\nabla J(0,0)| = \sqrt{8^2 + 10^2} \approx 12.8.
	\]}
\bigskip

\item Find the direction of minimum rate of change of $J$ at the point $\boldsymbol{\theta}^\intercal = (0,0)$. What is the rate of change in this direction?

\bigskip
\textcolor{red}{Theorem 11.2 tells us that the direction of minimum rate of change is in the direction of the negative gradient vector
	\[-\nabla J(0,0) = \begin{bmatrix}
	8 \\ 10
	\end{bmatrix}.
	\]
The rate of change in this direction is given by the directional first derivative $J'_\mathbf{v}(0,0)$, where $\mathbf{v}$ is the unit vector that points in the same direction as the negative gradient:
	\[\mathbf{v} = -\frac{\nabla J(0,0)}{|\nabla J(0,0)|}.
	\]
But then
	\[J'_\mathbf{v}(0,0) = \mathbf{v}^\intercal \nabla J(0,0) = -\frac{\nabla J(0,0)^\intercal \nabla J(0,0)}{|\nabla J(0,0)|} =- \frac{|\nabla J(0,0)|^2}{|\nabla J(0,0)|} = -|\nabla J(0,0)| = -\sqrt{8^2 + 10^2} \approx -12.8.
	\]}
\bigskip
\end{enumerate}







\prob Yet again, consider the function $J$ from Problem 5:
	\[J: \mathbb{R}^2 \to \mathbb{R}, \quad J(\boldsymbol{\theta}) = 2\theta_1^2 + 4\theta_1\theta_2 + 3\theta_2^2 -8\theta_1 - 10\theta_2 + 9.
	\]

Find and classify all extremizers of $J$.

\bigskip
\textcolor{red}{We first solve the stationarity equation $\nabla J(\theta_1,\theta_2)=\boldsymbol{0}$ for $\boldsymbol{\theta}$, which is
	\[\begin{bmatrix}
	4\theta_1 + 4\theta_2 -8 \\
	6\theta_2 + 4\theta_1-10
	\end{bmatrix} = \begin{bmatrix}
	0 \\ 0
	\end{bmatrix}.
	\]
The solution is $(\boldsymbol{\theta}^\star)^\intercal = (1,1)$. Then, we consider the Hessian matrix:
	\[\nabla^2 J(1,1) = \begin{bmatrix} 4 & 4 \\ 4 & 6 \end{bmatrix}.
	\]
Its eigenvalues are $5 \pm \sqrt{17}$, which are both positive. Hence it is positive definite, so by the Second Derivative Test, the point $\boldsymbol{\theta}^\star$ is a minimizer of $J$. (It is in fact the global minimizer.)}
\bigskip










\prob For the fifth time, consider the function $J$ from Problem 5:
	\[J: \mathbb{R}^2 \to \mathbb{R}, \quad J(\boldsymbol{\theta}) = 2\theta_1^2 + 4\theta_1\theta_2 + 3\theta_2^2 -8\theta_1 - 10\theta_2 + 9.
	\]
Find the directions of extreme curvature at the point $\boldsymbol{\theta} = (0,0)$. What are the curvatures in these directions?

\bigskip
\textcolor{red}{Theorem 11.4 tells us that the directions of extreme curvature are given by the eigenvectors $\mathbf{e}_1$ and $\mathbf{e}_2$ corresponding to the eigenvalues
	\[\lambda_1 = 5 - \sqrt{17} \approx 0.88 \quad \text{and} \quad \lambda_2 = 5 + \sqrt{17} \approx 9.12.
	\]
Using technology, we compute these eigenvectors
	\[\mathbf{e}_1 \approx \begin{bmatrix}
	-0.79 \\ 0.62
	\end{bmatrix}, \quad
	\mathbf{e}_2 \approx \begin{bmatrix}
	-0.62 \\ -0.79
	\end{bmatrix}.
	\]
The theorem also tells us that the curvatures are the eigenvalues themselves.}
\bigskip










\prob One last time, consider the function $J$ from Problem 5:
	\[J: \mathbb{R}^2 \to \mathbb{R}, \quad J(\boldsymbol{\theta}) = 2\theta_1^2 + 4\theta_1\theta_2 + 3\theta_2^2 -8\theta_1 - 10\theta_2 + 9.
	\]
Set $\mathbf{H} = \nabla^2J(0,0)$. Compute the spectrum $\sigma(\mathbf{H})$, the spectral radius $\rho( \mathbf{H})$, and the condition number $\kappa(\mathbf{H})$.

\bigskip
\textcolor{red}{We already computed the spectrum of the Hessian to be (approximately) $\{0.88, 9.12\}$. Thus, the spectral radius and condition number are given by
	\[\rho(\mathbf{H} ) \approx 9.12, \quad \kappa(\mathbf{H} ) \approx \frac{9.12}{0.88} \approx 10.36.
	\]}
\bigskip








\prob Consider the objective function
	\[J:\mathbb{R}^2 \to \mathbb{R}, \quad J(\boldsymbol{\theta}) = 2\theta_1^2 + 4\theta_2^2.
	\]


\begin{enumerate}
\item Write down the gradient descent update rule. Suppose the learning rate is $\alpha$, while the decay rate is $\beta=0$.

\bigskip
\textcolor{red}{We have
	\[ \nabla J(\boldsymbol{\theta}) = \mathbf{H} \boldsymbol{\theta},
	\]
where
	\[\mathbf{H} = \begin{bmatrix}
	4 & 0 \\ 0 & 8
	\end{bmatrix}.
	\]
Thus, the update rule is
	\[\boldsymbol{\theta} := \boldsymbol{\theta} - \alpha \mathbf{H} \boldsymbol{\theta}.
	\]
In the form of a recurrence relation, this is
	\[\boldsymbol{\theta}_{t+1} = \boldsymbol{\theta}_t - \alpha \mathbf{H} \boldsymbol{\theta}_t.
	\]}
\bigskip

\item Find a closed form expression for $\boldsymbol{\theta}_t$.

\bigskip
\textcolor{red}{We have
	\[\boldsymbol{\theta}_t = (I-\alpha \mathbf{H})^t \boldsymbol{\theta}_0 = \begin{bmatrix}
	(1-4\alpha)^t & 0 \\ 0 & (1-8\alpha)^t
	\end{bmatrix} \boldsymbol{\theta}_0
	\]
for all $t\geq 1$.}
\bigskip

\item Using your answer to (b), discuss convergence of the gradient descent algorithm.

\bigskip
\textcolor{red}{For all $t\geq 1$, we have
	\[(\boldsymbol{\theta}_t)_1 = (1-4\alpha)^t (\boldsymbol{\theta}_0)_1 \quad \text{and} \quad (\boldsymbol{\theta}_t)_2 = (1-8\alpha)^t (\boldsymbol{\theta}_0)_2.
	\]
Then the algorithm will converge to $\boldsymbol{\theta}^\star = (0,0)$ if and only if both $|1-4\alpha|<1$ and $|1-8\alpha|<1$. But this happens if and only if $0 < \alpha < 1/4$.}
\bigskip
\end{enumerate}










\prob Consider the stochastic objective function
	\[J:\mathbb{R}^2 \to \mathbb{R}, \quad J(\boldsymbol{\theta}) = \frac{1}{m} \sum_{i=1}^m \frac{1}{2}|\mathbf{x}_i - \boldsymbol{\theta}|^2
	\]
from class, where $\mathbf{x}_1,\mathbf{x}_2, \ldots, \mathbf{x}_m \in \mathbb{R}^2$ is an observed dataset.

\medskip
\begin{enumerate}
\item Compute the update rule in the batch gradient descent algorithm with learning rate $\alpha$ and decay rate $\beta$.

\bigskip
\textcolor{red}{Note that
	\[\frac{1}{2}|\mathbf{x} - \boldsymbol{\theta}|^2 = \frac{1}{2}(x_1 - \theta_1)^2 + \frac{1}{2}(x_2 - \theta_2)^2,
	\]
so
	\[\nabla J(\boldsymbol{\theta}) = \frac{1}{m} \sum_{i=1}^m \nabla_{\boldsymbol{\theta}} \left( |\mathbf{x}_i - \boldsymbol{\theta}|\right) = \frac{1}{m} \sum_{i=1}^m \begin{bmatrix}
	\theta_1- x_{i1} \\
	\theta_2 - x_{i2}
	\end{bmatrix} = \boldsymbol{\theta} - \bar{\mathbf{x}},
	\]
where $\bar{\mathbf{x}}$ is the empirical mean of the dataset. Thus, the update rule is
	\[\boldsymbol{\theta} := \boldsymbol{\theta} - \alpha(1 - \beta)^{t+1} (\boldsymbol{\theta} - \bar{\mathbf{x}}).
	\]
The recurrence relation is
	\[\boldsymbol{\theta}_{t+1} = \boldsymbol{\theta}_t - \alpha(1 - \beta)^{t+1} (\boldsymbol{\theta}_t - \bar{\mathbf{x}})
	\]
for $t\geq 0$.}
\bigskip

\item Assuming $\beta=0$, discuss convergence of the batch gradient descent algorithm.

\bigskip
\textcolor{red}{From part (a), we compute
	\[\boldsymbol{\theta}_t - \bar{\mathbf{x}} = (1-\alpha)^t(\boldsymbol{\theta}_0 - \bar{\mathbf{x}})
	\]
for all $t\geq 1$. Thus, the algorithm converges to the empirical mean $\mathbf{x}$ if the learning rate is $\alpha<1$.}
\bigskip
\end{enumerate}




\end{document}