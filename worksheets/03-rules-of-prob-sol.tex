\documentclass[12pt,reqno]{amsart}
\usepackage{./header, amssymb}

\hdr{Mathematical Statistics}{Chapter 3: Rules of probability}

\begin{document}


\bigskip
\prob

\begin{enumerate}
\item The chairs of an auditorium are to be labeled with an uppercase English letter followed by a positive integer not exceeding $100$. What is the largest number of chairs that can be labeled differently?

\bigskip
\textcolor{red}{Our task it to label the chairs in the auditorium. But to label a chair, we must do two things: First, choose an uppercase English letter, then chose a positive integer not exceeding 100. Since this procedure can be broken down into a sequence of two tasks, the Product Rule for Counting tells us that there are $26 \times 100 = 2{,}600$ ways to label the chairs.}
\bigskip

\item A student can choose a computer project from one of three lists. The three lists contain $23$, $15$, and $19$ possible projects, respectively. No project is on more than one list. How many possible projects are there to choose from?

\bigskip
\textcolor{red}{The procedure that the student must complete is to choose a project from the three lists. But because no project appears on more than one list, the Sum Rule for Counting tells us that there are $23+15+19 = 57$ possible projects.}
\end{enumerate}









\bigskip
\prob Suppose that there are eight runners in a race. The winner receives a gold medal, the second-place finisher receives a silver medal, and the third-place finisher receives a bronze medal. How many different ways are there to award these medals, if all possible outcomes of the race can occur and there are no ties?

\bigskip
\textcolor{red}{When selecting the three people who win a medal, the order matters. Hence we are looking for the number of $3$-permutations from a set of $8$ elements. The answer is then
	\[
	P^8_3 = \frac{8!}{5!} = 336.
	\]}
\bigskip







\prob How many poker hands of five cards can be dealt from a standard deck of 52 cards?

\bigskip
\textcolor{red}{We are asked to compute how many ways there are to select five cards from a set of 52 cards. Since order does not matter, these are combinations, not permutations. Thus there are
	\[
	\binom{52}{5} = \frac{52!}{5!47!} = 2,598,960
	\]
many poker hands.}
\bigskip







\bigskip
\prob Let $A$ and $B$ be two events in a sample space for which $P(A) = 2/3$, $P(B) = 1/6$, and $P(A \cap B) = 1/9$. What is $P(A \cup B)$?


\bigskip
\textcolor{red}{By the Sum Rule for Probability, we have
	\[
	P(A\cup B) = P(A) + P(B) - P(A\cap B) = 2/3 + 1/6 - 1/9 \approx 0.72.
	\]}















\bigskip
\prob Let $A$ and $B$ be two events in a sample space $S$ for which one knows that the probability that at least one of them occurs is $3/4$. What is the probability that neither $A$ nor $B$ occurs?

\bigskip
\textcolor{red}{The event that neither $A$ nor $B$ occurs is
	\[
	A^c \cap B^c,
	\]
where, as usual, we set $A^c = S\smallsetminus A$ and $B^c = S\smallsetminus B$. But this is the same as
	\[
	(A\cup B)^c.
	\]
Then
	\[
	P((A\cup B)^c) = 1 - P(A\cup B) = 1 - 3/4 = 1/4.
	\]}










\bigskip
\prob If two events, $A$ and $B$, are such that $P(A)=0.5$, $P(B)=0.3$, and $P(A\cap B)=0.1$ find the following:

\medskip
\begin{enumerate}
\item $P(A|B)$ \textcolor{red}{$=P(A\cap B)/P(B) = 0.1/0.3 = 1/3$}
\item $P(B|A)$ \textcolor{red}{$=P(A\cap B)/P(A) = 0.1/0.5 = 1/5$}
\item $P(A |A\cap B)$ \textcolor{red}{$=P(A\cap (A\cap B))/P(A\cap B) = 1$}
\end{enumerate}









\bigskip
\prob A survey of consumers in a particular community showed that $10\%$ were dissatisfied with plumbing jobs done in their homes. Half the complaints dealt with plumber $A$, who does $40\%$ of the plumbing jobs in the town.

\medskip
\begin{enumerate}
\item Identify an appropriate sample space $S$ and probability measure for this scenario.
    
\bigskip
\textcolor{red}{The sample space $S$ is the set of all consumers in the community. Since the problem statement deals with proportions (i.e., percentages of the population), evidently the probability measure is uniform. This means the probabilities \textit{are} the proportions.}
\bigskip

\item Find the probability that a consumer will obtain an unsatisfactory plumbing job, given that the plumber was $A$.
    
\bigskip
\textcolor{red}{Naturally, we let $A$ denote the event in $S$ consisting of all consumers who dealt with plumber $A$. We let $U$ denote the event in $S$ consisting of all dissatisfied consumers. We are told $P(U)=0.1$, $P(A| U) = 0.5$, and $P(A)=0.4$. Now, we are asked to compute:
	\[
	P(U|A) = \frac{P(U\cap A)}{P(A)} = \frac{P(A|U)P(U)}{P(A)} = \frac{0.5 \times 0.1}{0.4} = 0.125.
	\]}
\bigskip

\item Find the probability that a consumer will obtain a satisfactory plumbing job, given that the plumber was $A$.
    
\bigskip
\textcolor{red}{Let $U^c$ denote the complement $S\smallsetminus U$. Then we compute:
	\[
	P(U^c | A) = 1 - P(U|A) = 1 - 0.125 = 0.875.
	\]}
\end{enumerate}














\bigskip
\prob A fair die is rolled twice. $A$ is the event that the total sum of the rolls equals $4$, while $B$ is the event that at least one of the rolls is a 3. Are $A$ and $B$ independent events?

\bigskip
\textcolor{red}{The event $A$ consists of the pairs of rolls
	\[
	(3,1),(2,2),(1,3),
	\]
so $P(A) = 3/36$. On the other hand, the cardinality of $B$ is $36-25 = 11$ since there are $25$ pairs of rolls that do not contain a $3$ (use the Product Rule for Counting). Hence, $P(B) = 11/36$. But $P(A\cap B) = 2/36$, and since
	\[
	\frac{2}{36} \neq \frac{3}{36} \times \frac{11}{36},
	\]
we conclude that the event are dependent.}
\bigskip















\bigskip
\prob Suppose that there is a $1$ in $50$ chance of injury on a single skydiving attempt. Suppose that an individual makes 50 dives.

\medskip
\begin{enumerate}
\item Identify an appropriate sample space $S$ for this scenario.
    
\bigskip
\textcolor{red}{The sample space for this scenario is all $50$-tuples of answers to the following sequence of questions:
	\[
	(\text{injured on jump 1?}, \text{ injured on jump 2?}, \ldots, \text{ injured on jump 50?}).
	\]
Thus, a sample point is a $50$-tuple of a bunch of \textit{yes}'s and \textit{no}'s.}
\bigskip

\item Assume that the outcome of any one of the dives is independent of the others. A friend claims there is a $100\%$ chance of injury if the skydiver jumps $50$ times. Is your friend correct? Why or why not?
    
\bigskip
\textcolor{red}{The friend is \textit{wrong}. To make this formal, notice that  we want to compute the probability of the event $A$ where there is at least one \textit{yes} in the $50$-tuple of answers. However, to compute $P(A)$, it will actually be easier to compute $P(A^c)$, where the complement $A^c = S\smallsetminus A$ is the event that \textit{all} the answers are \textit{no}'s. Letting $B_n$ denote the event that the diver is \textit{not} injured on the $n$-th dive, we have
	\[
	A^c = B_1 \cap B_2 \cap \cdots \cap B_{50}.
	\]
But the $B_n$'s are all independent of each other, so we have
	\[
	P(A^c) = P(B_1) \times P(B_2) \times \cdots \times P(B_{50}) = (0.98)^{50} \approx 0.364.
	\]
Thus, the probability we actually want to compute is
	\[
	P(A) = 1 - P(A^c) \approx 0.636.
	\]}
\end{enumerate}

















\bigskip
\prob A ball is drawn at random from an urn containing one red and one white ball. If the white ball is drawn, it is put back into the urn. If the red ball is drawn, it is returned to the urn together with two more red balls. Then a second draw is made. What is the probability a red ball was drawn on both the first and the second draws?

\bigskip
\textcolor{red}{Let $W$ be the event that the first draw is a white ball, so that the complement $W^c = S\smallsetminus W$ is the event that the first raw is red. Let $A$ be the event that a red ball is drawn on \textit{both} the first and second draw. Then, by the Total Law of Probability, we have
	\[
	P(A) = P(A|W)P(W) + P(A|W^c)P(W^c).
	\]
Now, we have $P(W) = P(W^c) = 0.5$, and
	\[
	P(A|W) = 0 \quad \text{and} \quad P(A|W^c) = 0.75.
	\]
Thus,
	\[
	P(A) = 0 \times 0.5 + 0.75 \times 0.5 = 0.375.
	\]}























\bigskip
\prob Suppose that a test has been devised to detect a certain disease. Suppose that:

\begin{itemize}
\item The disease affects $0.1\%$ of the population.
\item The test does not produce any false negatives.
\item The test produces false positives at a rate of $5\%$.
\end{itemize}

Given that a randomly selected individual tests positive for the disease, what is the probability that they have it?

\bigskip
\textcolor{red}{Let $D$ be the event that a randomly selected person has the disease, and let $T$ be the event that the person tests positive. We are told that $P(D) = 0.001$. A false negative occurs when a person who \textit{does} have the disease tests negative; so, we know that
	\[
	P(T^c|D) =0,
	\]
which is the same as
	\[
	P(T|D) =1.
	\]
On the other hand, a false positive occurs when a person who \textit{does not} have the disease tests positive; so, we know that
	\[
	P(T|D^c) = 0.05.
	\]
Our goal is to compute $P(D|T)$. But, by Bayes' Theorem and the two-event version of Law of Total Probability, we have
	\begin{align*}
	P(D|T) &= \frac{P(T|D)P(D)}{P(T)} \\
	&= \frac{P(T|D)P(D)}{P(T|D)P(D) + P(T|D^c)P(D^c)} \\
	&= \frac{1\times 0.001}{1 \times 0.001 + 0.05 \times (1-0.001)} \\
	&\approx 0.012.
	\end{align*}}







\end{document}