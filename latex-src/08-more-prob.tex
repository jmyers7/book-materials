\documentclass[12pt,reqno]{amsart}
\usepackage{./header, amssymb}

\hdr{Mathematical Statistics}{Chapter 8: More probability theory}

\begin{document}

\bigskip

\prob Suppose that $X$ and $Y$ are random variables with the joint density function

	\[
	f(x,y) = \begin{cases}
	2xy + 0.5 & : 0 \leq x, y \leq 1, \\
	0 & : \text{otherwise}.
	\end{cases}
	\]
	
Compute the covariance of $X$ and $Y$.

\bigskip
\textcolor{red}{Using the Shortcut Formula for Covariance, we compute:
	\[
	\sigma_{XY} = E(XY) - E(X) E(Y).
	\]
But first, let's grab the expectations of $X$ and $Y$. To do this, we integrate out $y$ to get the density of $x$:
	\[f(x) = \int_{\mathbb{R}} f(x,y) \ \text{d} y = \int_0^1 \left( 2xy+0.5\right) \ \text{d}y = x + 0.5
	\]
for $0\leq x\leq 1$, and $f(x)=0$ otherwise. Then, we compute:
	\[E(X) = \int_{\mathbb{R}} x f(x) \ \text{d}x = \int_0^1 (x^2+0.5x) \ \text{d} x = \frac{7}{12}.
	\]
Now, if you look at the joint density function, you'll notice that it is symmetric in $x$ and $y$. This means that $E(Y) = 7/12$, as well. Finally, we compute the covariance from the shortcut formula:
	\begin{align*}
	\sigma_{XY} &= E(XY) - \frac{7^2}{12^2} \\
	&= \iint_{\mathbb{R}^2} xy f(x,y) \ \text{d}y \text{d}x - \frac{7^2}{12^2} \\
	&= \int_0^1 \int_0^1 (2x^2y^2 + 0.5xy) \ \text{d}y \text{d}x - \frac{7^2}{12^2} \\
	&= \frac{25}{72} - \frac{7^2}{12^2} \\
	&= \frac{1}{144} \\
	&\approx 0.007.
	\end{align*}}













\bigskip
\prob Suppose that $X$ and $Y$ are random variables with the joint density function

	\[
	f(x,y) = \begin{cases}
	3x & : 0 \leq y\leq x \leq 1, \\
	0 & : \text{otherwise}.
	\end{cases}
	\]
	
Compute the covariance of $X$ and $Y$.


\bigskip
\textcolor{red}{We follow the same strategy as the previous problem. First, we get the marginal densities:
	\[f(x) = \int_{\mathbb{R}} f(x,y) \ \text{d}y = \int_0^x 3x \ \text{d}y = 3x^2
	\]
for $0\leq x \leq 1$ and $f(x)=0$ otherwise; also:
	\[f(y) = \int_{\mathbb{R}} f(x,y) \ \text{d} x = \int_y^1 3x \ \text{d}x = \frac{3}{2}(1-y^2)
	\]
for $0\leq y \leq 1$. Then, we compute:
	\[E(X) = \int_{\mathbb{R}} x f(x) \ \text{d}x = \int_0^1 3x^3 \ \text{d}x = \frac{3}{4}
	\]
and
	\[E(Y) = \int_{\mathbb{R}} y f(y) \ \text{d}y = \frac{3}{2} \int_0^1 y (1-y^2) \ \text{d}y = \frac{3}{8}.
	\]
Finally, we compute
	\[E(XY) = \iint_{\mathbb{R}^2} xy f(x,y) \ \text{d}y\text{d}x = \int_0^1 \int_0^x 3x^2y \ \text{d}y\text{d}x = \frac{3}{10}
	\]
and hence
	\[\sigma_{XY} = E(XY) - E(X) E(Y) = \frac{3}{10} - \frac{3}{4} \cdot \frac{3}{8} = \frac{3}{160} \approx 0.019.
	\]}
	


\end{document}