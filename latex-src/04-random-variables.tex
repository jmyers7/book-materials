\documentclass[12pt,reqno]{amsart}
\usepackage{./header}

\hdr{Mathematical Statistics}{Chapter 4: Random variables, \textit{etc}.}

\begin{document}

\bigskip

\prob Let the sample space $S$ represent the current population of the United States. Suppose also that to every individual $s\in S$ we assign the state $X(s)$ in which they currently reside. So, for example, I would write
	\[
	X(\text{John}) = \text{New York}.
	\]
Does this function
	\[
	X: S \to \{\text{Alabama}, \text{Alaska},\ldots,\text{Wyoming}\}
	\]
qualify as a random variable on $S$? If not, how could you make it fit the template of a random variable?\vfill


















\bigskip
\prob Suppose that we flip a fair coin twice. As usual, we model the situation using a uniform probability space with sample space

	\[
	S = \{ HH, HT, TH, TT\}.
	\]

Describe some random variables on $S$.\vfill



















\bigskip

\prob Suppose that we toss a pair of fair six-sided dice. As usual, we model the situation using a uniform probability space with sample space

	\[
	S = \{(1,1),(1,2),\ldots,(5,6),(6,6)\}.
	\]

Describe some random variables on $S$.\vfill

























\newpage
\prob Let

	\[
	S = \{ (x,y) \in \mathbb{R}^2 : x^2+y^2 \leq 1\}
	\]

be the solid unit disk in $\mathbb{R}^2$. Consider drawing a point uniformly at random from $S$.

\medskip
\begin{enumerate}
\item Define an appropriate probability space that models this scenario.\vfill
\item Describe some random variables in this scenario.\vfill
\end{enumerate}



























\bigskip
\prob You and a friend play a game where you each toss a fair coin. If both coins land tails, you win $\$1$; if they both land heads, you win $\$2$; if the coins do not match (one lands a head, the other a tail), you lose $\$1$ (win $-\$1$).

\medskip
\begin{enumerate}
\item Describe an appropriate probability space that models the situation.\vfill
\item Let $X$ be the random variable which gives your winnings on a single play of the game. Describe the probability distribution of $X$.\vfill
\end{enumerate}























\newpage
\prob Five balls, numbered $1$, $2$, $3$, $4$, and $5$, are placed in an urn. Two balls are randomly selected from the five, and their numbers noted. (The order of the selected balls does not matter.)

\medskip
\begin{enumerate}
\item Describe an appropriate probability space that models the situation.\vfill
\item Let $X = $ the largest of the two numbers. Describe the probability distribution of $X$.\vfill
\item Let $Y = $ the sum of the two numbers. Describe the probability distribution of $Y$.\vfill
\end{enumerate}















\bigskip
\prob Let $S$ be the discrete probability space $S = \{1,2,3,4\}$ with probability function

	\[
	\begin{array}{c|c}
	s & p(s) \\ \hline
	1 & 1/3\\
	2 & 1/3\\
	3 & 1/6\\
	4 & 1/6
	\end{array}
	\]

Define the random variable $X:S\to \mathbb{R}$ by

	\[
	X(s) = \begin{cases}
	0 & : s = 1, 4, \\
	1 & : s=2, \\
	2 & : s=3.
	\end{cases}
	\]

Describe the probability distribution of $X$.\vfill












\newpage
\prob Which of the random variables in problems 5-7 are discrete? Which are continuous?\vfill















\bigskip
\prob A gas station operates two pumps, each of which can pump up to 10,000 gallons of gas in a month. The total amount of gas pumped at the station in a month is a continuous random variable $X$ (measured in 10,000 gallons) with a probability density function given by

	\[
	f(x) = \begin{cases}
	x & : 0 < x < 1, \\
	2-x & : 1 \leq x < 2, \\
	0 & : \text{otherwise}.
	\end{cases}
	\]

\medskip
\begin{enumerate}
\item Find the probability that the station will pump between 8000 and 12,000 gallons in a particular month.\vfill
\item Given that the station pumped more than 10,000 gallons in a particular month, find the probability that the station pumped more than 15,000 gallons during a month.\vfill
\end{enumerate}









\newpage
\prob The length of time to failure (in hundreds of hours) for a transistor is a random variable $X$ with
distribution function given by

	\[
	F(x) = \begin{cases}
	1 - e^{-x^2} & : x\geq 0, \\
	0 & : \text{otherwise}.
	\end{cases}
	\]

\medskip
\begin{enumerate}
\item Compute the quantile $Q(0.3)$.\vfill
\item Compute the density $f(x)$ of $X$.\vfill
\item Find the probability that the transistor operates for at least 200 hours.\vfill
\item Compute the conditional probability $P(X>1 \mid X\leq 2)$.\vfill
\end{enumerate}



















\bigskip
\prob Compute the expected value $E(X)$ of the random variable $X$ in problem 5.\vfill



























\bigskip
\prob Suppose $X$ is a discrete random variable distributed uniformly on its range
    
	\[
	\{1,2,\ldots,n\},
	\]

for some integer $n\geq 1$. Compute the mean value of $X$.\vfill













\newpage
\prob Let $X$ be the number of interviews that a student has prior to getting a job. Suppose that the probability function of $X$ is given by

	\[
	p(x) = \begin{cases}
	k/x^2 & : x=1,2,\ldots, \\
	0 & : \text{otherwise,}
	\end{cases}
	\]

where $k$ is a constant such that $\sum_{x=1}^\infty k/x^2 = 1$. Compute the mean $\mu_X$ (if it exists).\vfill



















\bigskip
\prob Let $X$ be a continuous random variable with density function
	
	\[
	f(x) = \begin{cases}
	(3/8)x^2 & : 0 \leq x \leq 2, \\
	0 & : \text{otherwise}.
	\end{cases}
	\]

Compute the mean value of $X$.\vfill














\bigskip
\prob The proportion of time per day that all checkout counters in a supermarket are busy is a random variable $X$ with density function

	\[
	f(x) = \begin{cases}
	cx^2(1-x)^4 & : 0 \leq x \leq 1, \\
	0 & : \text{otherwise}.
	\end{cases}
	\]

\medskip
\begin{enumerate}
\item Find the value of $c$ that makes $f(x)$ a valid density function.\vfill
\item Compute the expected value $E(X)$.\vfill
\end{enumerate}















\newpage
\prob Consider the random variables $X$ and $Y$ defined in lecture via the table:

	\[
	\begin{array}{c|ccc}
	s & X(s) & Y(s) & (X+Y)(s) \\ \hline
	1 & -1 & 0 & -1 \\
	2 & 1 & 2 & 3 \\
	3 & 3 & -1 & 2 \\
	4 & 0 & 3 & 3
	\end{array}
	\]

Suppose that the probability distribution on $S$ is uniform.

\medskip
\begin{enumerate}
\item Compute the probability distributions of $X$, $Y$, and $X+Y$.\vfill
\item Without me even telling you, I bet you can guess the definition of the pointwise product $XY$. List the outputs of $XY$, and compute its probability distribution.\vfill
\end{enumerate}
    











\bigskip
\prob Suppose that we have a random variable $X$ on the finite sample space $S = \{1,2,3,4,5\}$ with

	\[\begin{array}{c|c}
	s & X(s)  \\ \hline
	1 & 0  \\
	2 & \pi/2  \\
	3 & \pi \\
	4 & 3\pi/2 \\
	5 & 2\pi
	\end{array}
	\]

\medskip
\begin{enumerate}
\item Compute the random variable $\sin{(X)}$.\vfill
\item Assume that the (discrete) probability distribution on $S$ has probability function $p(s)$ with

	\[
	\begin{array}{c|c}
	s & p(s)  \\ \hline
	1 & 0  \\
	2 & 1/8  \\
	3 & 1/2 \\
	4 & 1/4 \\
	5 & 1/8
	\end{array}
	\]
Compute the probability distribution of $\sin(X)$.\vfill
\end{enumerate}
















\newpage
\prob Compute the expectation of the random variable $\sin(X)$ in the previous problem.\vfill













\bigskip
\prob Let $X$ be a continuous random variable with density function
	
	\[
	f(x) = \begin{cases}
	(3/8)x^2 & : 0 \leq x \leq 2, \\
	0 & : \text{otherwise}.
	\end{cases}
	\]

Compute the mean value of $2X^2+1$.\vfill















\bigskip
\prob Let $X$ be a random variable on the same sample space $S = \{1,2,3,4\}$ with the following values:

	\[
	\begin{array}{c|c}
	s & X(s)  \\ \hline
	1 & 0  \\
	2 & 1  \\
	3 & 0 \\
	4 & 3
	\end{array}
	\]

Supposing that the probability distribution on $S$ has probability function given by

	\[
	\begin{array}{c|c}
	s & p(s)  \\ \hline
	1 & 1/3  \\
	2 & 1/9  \\
	3 & 1/3 \\
	4 & 2/9
	\end{array}
	\]

compute the expectation of the random variable $X^2 + X + 2$.\vfill

















\newpage
\prob Suppose that $X$ is a discrete random variable with probability function

	\[
	\begin{array}{c|c}
	x & p(x) \\ \hline
	0 & 1/8   \\
	1 & 1/4   \\
	2 & 3/8  \\
	3 & 1/4 
	\end{array}
	\]

Compute the expectation, variance, and standard deviation of $X$.\vfill





















\bigskip
\prob Suppose that a single fair six-sided die is tossed once and we let $X$ be the number facing up. Find the expected value, variance, and standard deviation of $X$.\vfill





















\bigskip
\prob Suppose that $X$ is a continuous random variable with density function
	
	\[
	f(x) = \begin{cases}
	(3/2)x^2 + x & : 0\leq x \leq 1, \\
	0 & : \text{otherwise}.
	\end{cases}
	\]

Compute the expectation, variance, and standard deviation of $X$.\vfill














\newpage
\prob

\begin{enumerate}
\item For a certain random variable $X$ it is known that $E(X)=2$ and $V(X) = 3$. What is $E(X^2)$?\vfill
\item Let $X$ be a random variable with $E(X)=2$ and $V(X)=4$. Compute the expectation and variance of $3-2X$.\vfill
\end{enumerate}















\bigskip
\prob Approximately $10\%$ of the glass bottles coming off a production line have serious flaws in the glass. If two bottles are randomly selected, find the mean, variance, and standard deviation of the number of bottles that have serious flaws.\vfill




\end{document}